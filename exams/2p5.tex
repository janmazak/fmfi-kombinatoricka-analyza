\documentclass[a4paper, 12pt]{article}

\usepackage[utf8]{inputenc}
\usepackage[T1]{fontenc}
\usepackage[slovak]{babel}

\usepackage{amsmath, amsthm, amssymb}

\addtolength{\voffset}{-3cm}
\addtolength{\hoffset}{-1.3cm}
\addtolength{\textwidth}{2.6 cm}
\addtolength{\textheight}{5cm}

\parskip = 2mm
% \parindent = 0pt

\def\R{\mathbb R}
\def\Z{\mathbb Z}

\def\ogf{\overset{{\rm ogf}}{\longleftrightarrow}}

\title{Kombinatorická analýza, 2. časť}
\date{}
\pagestyle{empty}
\begin{document}

\begin{enumerate}
\item
Nech $a_0=0$, $a_1=-2$ a
$$
a_{n+2} -4 a_{n+1}+3a_n = 3\cdot 2^{n+2} \quad\hbox{pre každé $n\ge 0$}.
$$
Dokážte, že $(a_n)\ogf {16x^2-2x\over (1-x)(1-2x)(1-3x)}$ a nájdite explicitné vyjadrenie $a_n$.
\medskip\hrule

\item
O postupnosti $(b_n)_{n\ge 0}$ je známe, že $b_0=1$ a
$$
b_n = 1 + \sum_{k=1}^n (n-k)b_k\quad\hbox{pre každé $n\ge 1$}.
$$
Dokážte, že pre $n>0$ platí $b_n = F_{2n-1}$.
% R20
\medskip\hrule

\item
Odhadnite s absolútnou presnosťou $O(n^{-3})$ hodnotu
$$
\sum_{k\ge 0} ke^{-k/n^2}.
$$
\medskip\hrule

\item
Odhadnite s absolútnou presnosťou $O(1)$ hodnotu
$$
\sum_{k=1}^{2n} (-1)^kH_k.
$$
% vysledok je O(1) + ln {2\cdot 4\cdot \dots \cdot 2n\over 1\cdot 3\cdot \dots \cdot (2n-1)} = O(1) + ln {(2^n\cdot n!)^2\over (2n)!} = O(1) + {1\over 2}\ln n
\medskip\hrule

\item
Rozhodnite, či existuje kladná konštanta $c$ taká, že
$$
S_n = \sum_{k=0}^{\lfloor n/3\rfloor} {n-2k\choose k}\left({-4\over 27}\right)^k = \Theta(c^n).
$$
Ak existuje, nájdite ju. Ak neexistuje, nájdite čo najmenšiu konštantu $d$ takú, že $S_n = O(d^n)$.


\end{enumerate}


\end{document}

